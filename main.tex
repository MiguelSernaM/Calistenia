\documentclass{article}
\usepackage[utf8]{inputenc}
\usepackage[spanish]{babel}
\usepackage{listings}
\usepackage{graphicx}
\graphicspath{ {images/} }
\usepackage{cite}

\begin{document}

\begin{titlepage}
    \begin{center}
        \vspace*{1cm}
            
        \Huge
        \textbf{Desafió de instrucciones }
            
        \vspace{0.5cm}
        \LARGE
        parcial 1
            
        \vspace{1.5cm}
            
        \textbf{Miguel Angel Serna Montoya}
            
        \vfill
            
        \vspace{0.8cm}
            
        \Large
        Departamento de Ingeniería Electrónica y Telecomunicaciones\\
        Universidad de Antioquia\\
        Medellín\\
        Marzo de 2021
            
    \end{center}
\end{titlepage}

\tableofcontents

\section{Guía}
Usted tiene que seguir los pasos que se presentaran a continuación al pie de la letra, teniendo en cuenta cada detalle y sin la ayuda de nadie. Para realizar la actividad  necesitara dos tarjetas estilo identificación o tip del mismo tamaño y una hoja de papel de un tamaño carta o por el estilo. Durante toda la actividad solo puede usar una mano.

\section{Instrucciones a seguir} \label{contenido}

No olvide que no puede recibir ayuda de nadie y solo puede usar una mano, las tarjetas van a estar alineadas y centradas bajo la hoja de papel.
\subsection{Primer paso} \label{contenido}
retire cuidadosamente la hoja de papel, luego tome las tarjetas con la mano y levántelas de la mesa, luego trate de alinear los lados de ambas tarjetas para que parezcan una sola (en caso de que cambiaran de posición) pero sin que toquen nada más que su mano. En el desarrollo de toda la actividad trate de que las tarjetas mantengan esta posición, solo cámbielas cuando se le indique.

\subsection{Segundo paso} \label{contenido}
Manteniendo las tarjetas en su mano con su respectiva posición trate de poner la hoja de papel en un lugar plano.
\subsection{Tercer paso} \label{contenido}
Una vez tenga la hoja de papel sobre un lugar plano, Sin soltar las tarjetas usted va a poner el lado mas corto de estas verticalmente sobre el papel
\subsection{Cuarto paso} \label{contenido}
Sin soltar las tarjetas y manteniendo estas sobre el papel vas a poner la cara de estas mirando hacia la palma de tu mano y vas a poner el dedo índice sobre la mitad del lado pequeño superior haciendo presión hacia abajo
\subsection{Quinto paso} \label{contenido}
Manteniendo la posición anterior trate de que el dedo índice mantenga unidas las cartas por su parte superior, vamos a jalar lentamente la carta que esta mas cercana de la palma de nuestra mano con ayuda del los dedos pulgar y anular, veremos que se va formando una especie de pirámide o choza y trataremos de equilibrarla para que se mantenga en pie.
\subsection{Sexto paso} \label{contenido}
Una vez veamos que tenemos la pirámide en equilibrio retiramos la mano lentamente.

\end{document}